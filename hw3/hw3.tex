\documentclass[12pt,letterpaper]{article}
\usepackage{fullpage}
\usepackage[top=2cm, bottom=4.5cm, left=2.5cm, right=2.5cm]{geometry}
\usepackage{amsmath,amsthm,amsfonts,amssymb,amscd}
\usepackage{lastpage}
\usepackage{enumerate}
\usepackage{fancyhdr}
\usepackage{mathrsfs}
\usepackage{xcolor}
\usepackage{graphicx}
\usepackage{listings}
\usepackage{hyperref}
\usepackage{enumitem}
\makeatletter
\renewcommand*\env@matrix[1][*\c@MaxMatrixCols c]{%
  \hskip -\arraycolsep
  \let\@ifnextchar\new@ifnextchar
  \array{#1}}
\makeatother

\hypersetup{%
  colorlinks=true,
  linkcolor=blue,
  linkbordercolor={0 0 1}
}
 
\renewcommand\lstlistingname{Algorithm}
\renewcommand\lstlistlistingname{Algorithms}
\def\lstlistingautorefname{Alg.}

\lstdefinestyle{Python}{
    language        = Python,
    frame           = lines, 
    basicstyle      = \footnotesize,
    keywordstyle    = \color{blue},
    stringstyle     = \color{green},
    commentstyle    = \color{red}\ttfamily
}

\setlength{\parindent}{0.0in}
\setlength{\parskip}{0.05in}

% Edit these as appropriate
\newcommand\course{UCU Linear Algebra}
\newcommand\hwnumber{3}                  % <-- homework number
\newcommand\NetIDa{Ivan Prodaiko}

\pagestyle{fancyplain}
\headheight 35pt
\lhead{\NetIDa}
\chead{\textbf{\Large Homework \hwnumber}}
\rhead{\course \\ \today}
\lfoot{}
\cfoot{}
\rfoot{\small\thepage}
\headsep 1.5em

\begin{document}

\section*{Problem 1}

\begin{enumerate}[label=(\alph*)]
    \item
        $A = \begin{pmatrix}
            \dfrac{1}{2} & \dfrac{3}{2} \\
            \dfrac{1}{2} & -\dfrac{1}{2}
        \end{pmatrix}$
        
        matrix A has constant columns sum, so one eigenvalue is equal to its sum equal to 1.
    \item
        $A = \begin{pmatrix}
            1 & -2 \\
            -2 & 4
        \end{pmatrix}$
        
        $det(A) = 4 - 4 = 0$
        
        Thus 0 is an eigenvalue and matrix is singular. By trace rule $tr(A) = 5$.
        
        $\lambda_1 = 0\quad\lambda_2 = 5$
    \item
        $A = \begin{pmatrix}
            1 & 2 \\
            4 & 3
        \end{pmatrix}$
        
        Sum of each column is equal to one another, so one of eigenvalues is $\lambda_1 = 5$.
        
        By trace rule $tr(A) = 4$
        
        $\lambda_1 = 5\quad\lambda_ 2 = -1$
    \item
        $A = \begin{pmatrix}
            0 & 2 & 1 \\
            2 & 0 & 1 \\
            1 & 2 & 0
        \end{pmatrix}$
        Sum of each row = 3 thus $\lambda_1 = 3$
        
        $tr(A) = \lambda_1 + \lambda_2 + \lambda_3 = 0$
        
        $det(A) = \lambda_1\lambda_2\lambda_3 = 6$
        
        $\lambda_2 + \lambda_3 = -3$
        
        $\lambda_2\lambda_3 = 2$
        
        $\lambda_2 = -2\quad\lambda_3 = -1$
    \item
        $A = \begin{pmatrix}
            2 & 0 & 2 \\
            0 & 4 & 0 \\
            3 & 0 & 2
        \end{pmatrix}$
        
        tr(A) = 2 + 4 + 2 = 8
        
        det(A) = -8
        
        It is easy to see, that if we subtract 4I from A we will get singular matrix, thus we could claim that 4 is an eigenvalue.
        
        $\lambda_1 = 4$
        
        $\lambda_2 + \lambda_3 = 8 - \lambda_1 = 4$
        
        $\lambda_2\lambda_3 = -2$
        
        $\lambda^2 - 4\lambda - 2 = 0$
        
        $D = 16 - 4 (-2) 1 = 32$
        
        $\lambda_2 = 2 + 2\sqrt{2}\quad\lambda_3 = 2 - 2\sqrt{2}$
\end{enumerate}

\section*{Problem 2}
    \begin{enumerate}[label=(\alph*)]
        \item
        $A = \begin{pmatrix}
            2 & 3 \\
            2 & 1
        \end{pmatrix}$
        
        $\lambda_1 = 4\quad\lambda_2 = -1$
        
        $B_1 =  \begin{pmatrix} 2 & 3 \\ 2 & 1 \end{pmatrix} - 
                \begin{pmatrix} 4 & 0 \\ 0 & 4 \end{pmatrix} = 
                \begin{pmatrix} -2 & 3 \\ 2 & -3 \end{pmatrix}$
                
        $B_2 =  \begin{pmatrix} 2 & 3 \\ 2 & 1 \end{pmatrix} - 
                \begin{pmatrix} -1 & 0 \\ 0 & -1 \end{pmatrix} = 
                \begin{pmatrix} 3 & 3 \\ 2 & 2 \end{pmatrix}$       
                
        $\begin{pmatrix}[cc|c]
          -2 & 3 & 0\\
          2 & -3 & 0
        \end{pmatrix} = -2x_1 + 3x_2 = 0$
        
        $-2x_1 = -3x_2$
        
        $x_1 = \dfrac{3}{2}x_2$
        
        $x_1 = \dfrac{3}{2}\quad x_2 = 1$
        
        $\vec{v_1} = \begin{pmatrix} \dfrac{3}{2} \\ 1 \end{pmatrix}$
        
        $\begin{pmatrix}[cc|c]
          3 & 3 & 0\\
          2 & 2 & 0
        \end{pmatrix} = 3x_1 + 3x_2 = 0$
        
        $3x_1 = -3x_2$
        
        $x_1 = -x_2$
        
        $x_1 = -1\quad x_2 = 1$
        
        $\vec{v_2} = \begin{pmatrix} -1 \\ 1 \end{pmatrix}$
        \item
        
        $A = \begin{pmatrix}
            -2 & 2 & 3 \\
            -2 & 3 & 2 \\
            -4 & 2 & 5
        \end{pmatrix}$
        
        $tr(A) = 6 \quad det(A) = 6$
        
        $\lambda_1 = 3$ because all row sums are equal to $3$
        
        $\lambda_2\lambda_3 = \dfrac{6}{3} = 2$
        
        $\lambda_2 + \lambda_3 = 6 - \lambda_1 = 3$
        
        $\lambda_2 = 2\quad\lambda_3 = 1$
        
        $\begin{pmatrix}
            -2 & 2 & 3 \\
            -2 & 3 & 2 \\
            -4 & 2 & 5
        \end{pmatrix} - \begin{pmatrix}
            3 & 0 & 0 \\
            0 & 3 & 0 \\
            0 & 0 & 3
        \end{pmatrix} = \begin{pmatrix}
            -5 & 2 & 3 \\
            -2 & 0 & 2 \\
            -4 & 2 & 2
        \end{pmatrix} => \begin{pmatrix}
            1 & -\dfrac{2}{5} & -\dfrac{3}{5} \\
            0 & 1 & -1 \\
            0 & 0 & 0
        \end{pmatrix}$
        
        $x_2 = x_3 = 1\quad x_1=\dfrac{2}{5}+\dfrac{3}{5}$
        
        $\vec{v_1} = \begin{pmatrix}
            1\\
            1\\
            1
        \end{pmatrix}$
        
        $\begin{pmatrix}
            -2 & 2 & 3 \\
            -2 & 3 & 2 \\
            -4 & 2 & 5
        \end{pmatrix} - \begin{pmatrix}
            2 & 0 & 0 \\
            0 & 2 & 0 \\
            0 & 0 & 2
        \end{pmatrix} = \begin{pmatrix}
            -4 & 2 & 3 \\
            -2 & 1 & 2 \\
            -4 & 2 & 3
        \end{pmatrix} => \begin{pmatrix}
            1 & -\dfrac{1}{2} & 0 \\
            0 & 0 & 1 \\
            0 & 0 & 0
        \end{pmatrix}$
        
        $x_1 = x_2 = \dfrac{1}{2}\quad x_3=0$
        
        $\vec{v_2} = \begin{pmatrix}
            \dfrac{1}{2}\\
            1\\
            0
        \end{pmatrix}$
        
        $\begin{pmatrix}
            -2 & 2 & 3 \\
            -2 & 3 & 2 \\
            -4 & 2 & 5
        \end{pmatrix} - \begin{pmatrix}
            1 & 0 & 0 \\
            0 & 1 & 0 \\
            0 & 0 & 1
        \end{pmatrix} = \begin{pmatrix}
            -3 & 2 & 3 \\
            -2 & 2 & 2 \\
            -4 & 2 & 4
        \end{pmatrix} => \begin{pmatrix}
            1 & 0 & -1 \\
            0 & 1 & 0 \\
            0 & 0 & 0
        \end{pmatrix}$
        
        $x_1 = x_3 = 1\quad x_2=0$
        
        $\vec{v_3} = \begin{pmatrix}
            1\\
            0\\
            1
        \end{pmatrix}$
        
        $A^2, A^{100}$:
        \begin{enumerate}
            \item
                $\lambda_1 = 16 \quad \lambda_2 = 1 \quad\quad\quad\quad\quad\quad \lambda_1 = 4^{100} \quad \lambda_2 = 1$
            \item
                $\lambda_1 = 9 \quad \lambda_2 = 4 \quad \lambda_3 = 1 \quad\quad\quad \lambda_1 = 3^{100} \quad \lambda_2 = 2^{100} \quad \lambda_3 = 1$
        \end{enumerate}
        $Av = \lambda v$
        
        $A^2v = \lambda^2v$
        
        So, with increasing of power eigenvalue would change accordingly while eigenvectors would stay the same
        
        $A^{-1}$:
        $A^{-1}Av = A^{-1}\lambda v$
        
        $v = A^{-1}\lambda v$
        
        $A^{-1}v = \dfrac{1}{\lambda} v$
        
        So, as expected, eigenvalues will change, while eigenvectors would stay the same.
        
        $e^{tA}$:
        
        $Av = \lambda v$
        
        $e^{tA}v = e^{t\lambda}v$
    \end{enumerate}
    
\section*{Problem 3}
If A $n$ x $n$ matrix has $n$ distinct eigenvalues, than $D$ is equal to matrix with eigenvalues of $A$ on its main diagonal.
\begin{enumerate}[label=(\alph*)]
    \item
        $A = \begin{pmatrix}
            2 & 3 \\
            2 & 1
        \end{pmatrix}$
        
        $\lambda_1 = 4\quad\lambda_2 = -1$
        
        $\vec{v_1} = \begin{pmatrix} \dfrac{3}{2} \\ 1 \end{pmatrix}$
        \quad
        $\vec{v_2} = \begin{pmatrix} -1 \\ 1 \end{pmatrix}$
        
        $D = \begin{pmatrix} 
            4 & 0 \\
            0 & -1
        \end{pmatrix}$
        
        Knowing that $AP = PD$ lets check our results.
        
        $P = \begin{pmatrix} 
            \dfrac{3}{2} & -1 \\ 
            1 & 1 
        \end{pmatrix}$
        
        $AP = \begin{pmatrix}
            2 & 3 \\
            2 & 1
        \end{pmatrix}
        \begin{pmatrix} 
            \dfrac{3}{2} & -1 \\ 
            1 & 1 
        \end{pmatrix} = 
        \begin{pmatrix}
            6 & 1 \\
            4 & -1
        \end{pmatrix}$
        
        $PD = \begin{pmatrix} 
            \dfrac{3}{2} & -1 \\ 
            1 & 1 
        \end{pmatrix}
        \begin{pmatrix} 
            4 & 0 \\
            0 & -1
        \end{pmatrix} = \begin{pmatrix}
            6 & 1 \\
            4 & -1
        \end{pmatrix}$
        
        $D$ is correctly found.
    \item
        $A = \begin{pmatrix} 
            2 & 1 \\ 
            5 & -2 
        \end{pmatrix}$
        
        $det(A) = \lambda_1\lambda_2 = -4 - 5 = -9$
        
        $tr(A) = \lambda_1+\lambda_2 = 0$
        
        $\lambda_1 = 3\quad \lambda_2 = -3$
        
        $D = \begin{pmatrix} 
            3 & 0 \\ 
            0 & -3 
        \end{pmatrix}$
        
        $v_1 = \begin{pmatrix} 1 \\ 1 \end{pmatrix}$
        \quad
        $v_2 = \begin{pmatrix} -\dfrac{1}{5} \\ 1 \end{pmatrix}$
        
        $P = \begin{pmatrix} 
            1 & -\dfrac{1}{5} \\ 
            1 & 1 
        \end{pmatrix}$
        
        $AP = \begin{pmatrix}
            2 & 1 \\ 
            5 & -2
        \end{pmatrix}
        \begin{pmatrix} 
            1 & -\dfrac{1}{5} \\ 
            1 & 1 
        \end{pmatrix} = 
        \begin{pmatrix}
            3 & \dfrac{3}{5} \\ 
            3 & -3
        \end{pmatrix}$
        
        $PD = 
        \begin{pmatrix} 
            1 & -\dfrac{1}{5} \\ 
            1 & 1 
        \end{pmatrix}
        \begin{pmatrix} 
            3 & 0 \\ 
            0 & -3
        \end{pmatrix} = 
        \begin{pmatrix}
            3 & \dfrac{3}{5} \\ 
            3 & -3
        \end{pmatrix}$
    \item
        $A = \begin{pmatrix}
            1 & 3 \\
            2 & 2
        \end{pmatrix}$
        
        $\lambda_1 = 4\quad\lambda_2 = -1$
    
    \item
    $A = \begin{pmatrix} 
            2 & 1 \\ 
            5 & -2 
        \end{pmatrix}$
        
    Sums of rows = 3, thus $\lambda_1 = 3$
    
    $tr(A) = 6$
    
    $\lambda_2 = 3$
    
    $P = \begin{pmatrix} 
            1 & 1 \\ 
            1 & 1 
        \end{pmatrix}$
        
    $P$ doesn't form a basis. $A$ is not diagonalizable.
\end{enumerate}

\section*{Problem 4}

 $A = \begin{pmatrix}
            3 & -2 & 0 \\
            0 & 3 & 0 \\
            0 & 1 & 3
        \end{pmatrix}$
        
$\lambda_{1, 2, 3} = 3$

Lets find eigenvectors

$A = \begin{pmatrix}
            3 & -2 & 0 \\
            0 & 3 & 0 \\
            0 & 1 & 3
        \end{pmatrix} - \begin{pmatrix}
            3 & 0 & 0 \\
            0 & 3 & 0 \\
            0 & 0 & 3
        \end{pmatrix} = \begin{pmatrix}
            0 & -2 & 0 \\
            0 & 0 & 0 \\
            0 & 1 & 0
        \end{pmatrix}$

After row reduction we have:

$\begin{pmatrix}
            0 & 1 & 0 \\
            0 & 0 & 0 \\
            0 & 0 & 0
        \end{pmatrix}$
        
This means, that our eigenvectors are in the form $v_{1, 2} = \begin{pmatrix}
            a \\
            0 \\
            b
        \end{pmatrix}$
        
But in order to form matrix P we must complete the basis, so we need to find generalized eigenvector $(A - \lambda I)v_k = v_{k-1}$

$\begin{pmatrix}
            0 & -2 & 0 \\
            0 & 0 & 0 \\
            0 & 1 & 0
        \end{pmatrix}\begin{pmatrix}
            x \\
            y \\
            z
        \end{pmatrix} = \begin{pmatrix}
            a \\
            0 \\
            b
        \end{pmatrix}$
        
So, as we could see, we need to find such a vector, that contains the $y$ value non zero (to increase rank). The simplest vector possible in this case is:

$v_3 = \begin{pmatrix}
            0 \\
            1 \\
            0
        \end{pmatrix}$
        
$v_ 2 = \begin{pmatrix}
            0 & -2 & 0 \\
            0 & 0 & 0 \\
            0 & 1 & 0
        \end{pmatrix}\begin{pmatrix}
            0 \\
            1 \\
            0
        \end{pmatrix} = \begin{pmatrix}
            -2 \\
            0 \\
            1
        \end{pmatrix}$

$v_ 1 = \begin{pmatrix}
            0 & -2 & 0 \\
            0 & 0 & 0 \\
            0 & 1 & 0
        \end{pmatrix}\begin{pmatrix}
            -2 \\
            0 \\
            1
        \end{pmatrix} = \begin{pmatrix}
            0 \\
            0 \\
            0
        \end{pmatrix}$

With such a vector $v_1$ we aren't able to form a basis. So, lets change it (we are allowed to do this, as far as we are restricted only by form of a vector $\begin{pmatrix}
            a \\
            0 \\
            b
        \end{pmatrix}$).
        
We will take $\begin{pmatrix}
            2 \\
            0 \\
            0
        \end{pmatrix}$ in order to eliminate $D_{13}$ entry.
        
Thus we have $P = \begin{pmatrix}
            2 & -2 & 0 \\
            0 & 0 & 1 \\
            0 & 1 & 0
        \end{pmatrix}$
        
$P^{-1}AP = \begin{pmatrix}
            \dfrac{1}{2} & 0 & 1 \\
            0 & 0 & 1 \\
            0 & 1 & 0
        \end{pmatrix}\begin{pmatrix}
            3 & -2 & 0 \\
            0 & 3 & 0 \\
            0 & 1 & 3
        \end{pmatrix}\begin{pmatrix}
            2 & -2 & 0 \\
            0 & 0 & 1 \\
            0 & 1 & 0
        \end{pmatrix} = \begin{pmatrix}
            3 & 0 & 0 \\
            0 & 3 & 1 \\
            0 & 0 & 3
        \end{pmatrix}$

\section*{Problem 5}
\begin{enumerate}[label=(\alph*)]
    \item
    $\vec{x_n} = A\vec{x_{n-1}}$
    
    $\vec{x_n} = PD^{n}P^{-1}\vec{x_0} = A^n\vec{x_0}$
    
    $\vec{x_n} = \begin{pmatrix}
            \dfrac{1}{2} & 0 & 1 \\
            0 & 0 & 1 \\
            0 & 1 & 0
        \end{pmatrix}\begin{pmatrix}
            3^n & 0 & 0 \\
            0 & 3^n & 0 \\
            0 & 0 & 3^n
        \end{pmatrix}\begin{pmatrix}
            2 & -2 & 0 \\
            0 & 0 & 1 \\
            0 & 1 & 0
        \end{pmatrix} = \begin{pmatrix}
            3^n\\
            3^n\\
            3^n
        \end{pmatrix}$
\end{enumerate}

\section*{Problem 7}
\begin{enumerate}[label=(\alph*)]
    \item 
    In order to have the same eigenvalues, $AB$ and $BA$ must have the same characteristic polynomial, means $AB$ and $BA$ must be similar.
    
    If A is invertible then $A^{-1}(AB)A = BA$, so $AB$ and $BA$ are similar.
    \item
    $tr(AB) = tr(BA)$
    
    $det(AB) = det(BA)$
    
    So, that means they have the same characteristic polynomial, thus their eigenvalues are the same.
\end{enumerate}

\section*{Problem 8}
\begin{enumerate}[label=(\alph*)]
    \item
    
    $A^2 = (PDP^{-1})(PDP^{-1}) = PD(P^{-1}P)DP = PD^2P^{-1}$
    
    $A^k = PD^kP^{-1}$
    
    \item
    
    Involutory matrix is the matrix that is equal to its own inverse.
    $n$ x $n$ matrix is diagonalizable $\iff$ it has n distinct eigenvalues.
    As far as entries of such a matrix are only 1 and -1 than the eigenvalues will be \in $\{1, -1\}$.
    
    So $A^2 = P^{-1}D^2P = P^{-1}IP = P^{-1}P = I$
    
    $A^2 = I$
\end{enumerate}

\section*{Problem 9}
$B = \begin{pmatrix}
            0 & 0 & 1 \\
            0 & 1 & 0 \\
            1 & 0 & 0
        \end{pmatrix}$
        
$B^2 = \begin{pmatrix}
            0 & 0 & 1 \\
            0 & 1 & 0 \\
            1 & 0 & 0
        \end{pmatrix}\begin{pmatrix}
            0 & 0 & 1 \\
            0 & 1 & 0 \\
            1 & 0 & 0
        \end{pmatrix} = \begin{pmatrix}
            1 & 0 & 0 \\
            0 & 1 & 0 \\
            0 & 0 & 1
        \end{pmatrix}$
        
$det(B) = -1 \quad tr(B) = 1$

$\lambda_1 = -1\quad\lambda_2 = 1\quad\lambda_3 = 1$

$D = \begin{pmatrix}
            -1 & 0 & 0 \\
            0 & 1 & 0 \\
            0 & 0 & 1
        \end{pmatrix}$
        
$A_1 = \begin{pmatrix}
            0 & 0 & 1 \\
            0 & 1 & 0 \\
            1 & 0 & 0
        \end{pmatrix} - \begin{pmatrix}
            -1 & 0 & 0 \\
            0 & -1 & 0 \\
            0 & 0 & -1
        \end{pmatrix} = \begin{pmatrix}
            1 & 0 & 1 \\
            0 & 2 & 0 \\
            1 & 0 & 1
        \end{pmatrix}$

Lets convert it to RREF.

$\begin{pmatrix}
            1 & 0 & 1 \\
            0 & 2 & 0 \\
            0 & 0 & 0
        \end{pmatrix}$, so $\vec{v_1} = \begin{pmatrix}
            -1 \\
            0 \\
            1
        \end{pmatrix}$
        
$A_2 = \begin{pmatrix}
            -1 & 0 & 1 \\
            0 & 0 & 0 \\
            1 & 0 & -1
        \end{pmatrix}$, so $\vec{v_2} = \begin{pmatrix}
            1 \\
            0 \\
            1
        \end{pmatrix}$
        
After all, we need generalized vector $v_3 = \begin{pmatrix}
            0 \\
            1 \\
            0
        \end{pmatrix}$
        
$P = \begin{pmatrix}
            -1 & 1 & 0 \\
            0 & 0 & 1 \\
            1 & 1 & 0
        \end{pmatrix}$

$P^{-1}BP = \begin{pmatrix}
            -\dfrac{1}{2} & 0 & \dfrac{1}{2} \\
            \dfrac{1}{2} & 0 & \dfrac{1}{2} \\
            0 & 1 & 0
        \end{pmatrix}\begin{pmatrix}
            0 & 0 & 1 \\
            0 & 1 & 0 \\
            1 & 0 & 0
        \end{pmatrix}\begin{pmatrix}
            -1 & 1 & 0 \\
            0 & 0 & 1 \\
            1 & 1 & 0
        \end{pmatrix} = \begin{pmatrix}
            -1 & 0 & 0 \\
            0 & 1 & 0 \\
            0 & 0 & 1
        \end{pmatrix}$

$P^{-1}BP$ has diagonal form.

\section*{Problem 11}
$g_n = 3g_{n-1} - g_{n-2} - g_{n-3}$

Having $x_n = \begin{pmatrix}
            g_{n-2} \\
            g_{n-1} \\
            g_n
        \end{pmatrix}$

$Ax_n = \begin{pmatrix}
            g_{n-2} \\
            g_{n-1} \\
            3g_{n-1} - g_{n-2} - g_{n-3}
        \end{pmatrix} = x_{n+1}$
        
Thus we could write $A = \begin{pmatrix}
            0 & 1 & 0 \\
            0 & 0 & 1 \\
            -1 & -1 & 3
        \end{pmatrix} = x_{n+1}$

As we have sums of each row equal, than $1$ is an eigenvalue.

$det(A) = -1\quad tr(A) = 3$, thus $\lambda_2\lambda_3 = -1\quad \lambda_2 + \lambda_3 = 2$

$\lambda_2 = 1 + \sqrt{2}$ $\lambda_3 = 1 - \sqrt{2}$

As far as $A$ has 3 distinct eigenvalues we could determine matrix $D$.

$D = \begin{pmatrix}
            1 & 0 & 0 \\
            0 & 1 + \sqrt{2} & 0 \\
            0 & 0 & 1 - \sqrt{2}
        \end{pmatrix}$

Skipping eigenvectors calculation...

$P = \begin{pmatrix}
            1 & 1 & \sqrt{17} \\
            1 & \sqrt{2} - 1 & -\sqrt{2} - 1 \\
            1 & 1 & 1
        \end{pmatrix}$
        
\section*{Problem 12}

$A(x_1, x_2, x_3) = (x_3, x_1, x_2)$

Transformation a is a simple permutation matrix $\begin{pmatrix}
            0 & 1 & 0 \\
            0 & 0 & 1 \\
            1 & 0 & 0
        \end{pmatrix}$
        
$tr(A) = 0\quad det(A) = 1$ Having same sums of rows and columns we could say that $\lambda_1 = 1$
        
\textbf{Eigenvalues}:

$\lambda_1 = 1 \quad \lambda_2 = \dfrac{1}{2}(-1 - i\sqrt{3}) \quad \lambda_3 = \dfrac{1}{2}(-1 + i\sqrt{3})$

\textbf{Eigenvectors}:

$\vec{v_1} = \begin{pmatrix}
            1\\
            1\\
            1
        \end{pmatrix} \quad \vec{v_2} = \begin{pmatrix}
            \dfrac{1}{2}(-1 + i\sqrt{3})\\
            -1 + \dfrac{1}{2}(1 - i\sqrt{3})\\
            1
        \end{pmatrix} \quad \vec{v_3} = \begin{pmatrix}
            \dfrac{1}{2}(-1 - i\sqrt{3})\\
            -1 + \dfrac{1}{2}(1 + i\sqrt{3})\\
            1
        \end{pmatrix}$

\section*{Problem 14}
\begin{enumerate}[label=(\alph*)]
    \item 
    $det(A - \lambda I) = 0\quad det(B - \lambda I) = 0$
    
    $det(B - \lambda I) = det[P^{-1}(A - \lambda I) P] = det(P^{-1})det(A - \lambda I)det(P)$
    
    Since $det(P^{-1})det(P) = det(P^{-1}P) = det(I) = 1$
    
    $det(B - \lambda I) = det(A - \lambda I)$
    \item
    From assumption we know that $A$ and $D$ are similar, this means they share the same characteristic polynomial. Since triangular matrix has eigenvalues on its main diagonal, we could say than $D$, diagonal matrix, consists only of eigenvalues of $A$.
    \item
    \textbf{Determinant}:
    
    If $A$ is $n$ x $n$ matrix and it is diagonalziable, than it has $n$ distinct eigenvalues.
    
    $det(A - \lambda I) = (\lambda_1 - \lambda)(\lambda_2 - \lambda)\dots(\lambda_n - \lambda)$
    
    The formula above holds for every $\lambda$, so assume $\lambda = 0$, thus $det(A) = \lambda_1\lambda_2\dots\lambda_n$
    
    \textbf{Trace}:
    
    In matrix $D$ that is the matrix consisting of all the eigenvalues of matrix $A$ the $tr(D)$ is the sum of all eigenvalues.
    
    $A = P^{-1}DP$
    
    $D = tr(P^{-1}AP) = tr((AP)) = tr((AP)P^{-1}) = tr(A)$
    \item
    \textbf{Product}:
    
    The formula holds as is.
    
    \textbf{Sum}:
    Using the fact that Jordan form of matrix $A$ and matrix $A$ itself share the same characteristic polynomial.
    
    $0 = \prod_{i=1}^n (\lambda - \lambda_i) = \lambda^n - \lambda^{n - 1}\sum_{i=1}^n\lambda_i + \dots + (-1)^n \prod_{i=1}^n \lambda_i$
    
    There are $n$ terms containing a power $\lambda_{n-1}$ in the determinant expansion: $−A_{11}\lambda_{n−1}\dots−A_{11}\lambda_{n-1}$. Collecting these terms, we get that the coefficient associated with $\lambda_{n-1}$ in the characteristic polynomial. 
    We get $trace(A) = \sum_{i=1}^n\lambda_i$.
\end{enumerate}

\end{document}
